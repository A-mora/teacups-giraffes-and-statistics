\documentclass[]{article}
\usepackage{lmodern}
\usepackage{amssymb,amsmath}
\usepackage{ifxetex,ifluatex}
\usepackage{fixltx2e} % provides \textsubscript
\ifnum 0\ifxetex 1\fi\ifluatex 1\fi=0 % if pdftex
  \usepackage[T1]{fontenc}
  \usepackage[utf8]{inputenc}
\else % if luatex or xelatex
  \ifxetex
    \usepackage{mathspec}
  \else
    \usepackage{fontspec}
  \fi
  \defaultfontfeatures{Ligatures=TeX,Scale=MatchLowercase}
\fi
% use upquote if available, for straight quotes in verbatim environments
\IfFileExists{upquote.sty}{\usepackage{upquote}}{}
% use microtype if available
\IfFileExists{microtype.sty}{%
\usepackage{microtype}
\UseMicrotypeSet[protrusion]{basicmath} % disable protrusion for tt fonts
}{}
\usepackage[margin=1in]{geometry}
\usepackage{hyperref}
\hypersetup{unicode=true,
            pdftitle={speed-and-distance},
            pdfborder={0 0 0},
            breaklinks=true}
\urlstyle{same}  % don't use monospace font for urls
\usepackage{graphicx,grffile}
\makeatletter
\def\maxwidth{\ifdim\Gin@nat@width>\linewidth\linewidth\else\Gin@nat@width\fi}
\def\maxheight{\ifdim\Gin@nat@height>\textheight\textheight\else\Gin@nat@height\fi}
\makeatother
% Scale images if necessary, so that they will not overflow the page
% margins by default, and it is still possible to overwrite the defaults
% using explicit options in \includegraphics[width, height, ...]{}
\setkeys{Gin}{width=\maxwidth,height=\maxheight,keepaspectratio}
\IfFileExists{parskip.sty}{%
\usepackage{parskip}
}{% else
\setlength{\parindent}{0pt}
\setlength{\parskip}{6pt plus 2pt minus 1pt}
}
\setlength{\emergencystretch}{3em}  % prevent overfull lines
\providecommand{\tightlist}{%
  \setlength{\itemsep}{0pt}\setlength{\parskip}{0pt}}
\setcounter{secnumdepth}{0}
% Redefines (sub)paragraphs to behave more like sections
\ifx\paragraph\undefined\else
\let\oldparagraph\paragraph
\renewcommand{\paragraph}[1]{\oldparagraph{#1}\mbox{}}
\fi
\ifx\subparagraph\undefined\else
\let\oldsubparagraph\subparagraph
\renewcommand{\subparagraph}[1]{\oldsubparagraph{#1}\mbox{}}
\fi

%%% Use protect on footnotes to avoid problems with footnotes in titles
\let\rmarkdownfootnote\footnote%
\def\footnote{\protect\rmarkdownfootnote}

%%% Change title format to be more compact
\usepackage{titling}

% Create subtitle command for use in maketitle
\newcommand{\subtitle}[1]{
  \posttitle{
    \begin{center}\large#1\end{center}
    }
}

\setlength{\droptitle}{-2em}

  \title{speed-and-distance}
    \pretitle{\vspace{\droptitle}\centering\huge}
  \posttitle{\par}
    \author{}
    \preauthor{}\postauthor{}
    \date{}
    \predate{}\postdate{}
  

\begin{document}
\maketitle

\section{\texorpdfstring{What does the `mean'
mean?}{What does the mean mean?}}\label{what-does-the-mean-mean}

Let's say you're interested in how dogs from Florida and Georgia differ
by weight.

\subsubsection{Sampling}\label{sampling}

Since we can't take the weight of EVERY dog in these these two states,
we have to rely on taking the weights of randomly selected groups of
dogs. Let's weigh 100 different dogs in each state.

\subsubsection{Taking the mean}\label{taking-the-mean}

It'd be really hard to eyeball 200 different numbers and get a sense for
how these two dogs differ. We need to summarize these weights into a
single number per group that will be meaningful-- { the mean }!

The first thing might do is find the mean weight of group. The mean is
the average. It is the sum of all of the weights divided by the number
of observations (\# of dogs)

We'll use this symbol to represent the mean of each sample:
\(\color{blue}{\bar{x}}\) (read ``x-bar''), and is calculated with the
following equation:

\(\text(1)\)
\[\Large \color{blue}{\bar{x}} = \frac{x_1 + x_2 + ... + x_n}{n}\]

To make this more efficient, instead of writing out all the ``x's'' for
each observation, we can use the sigma symbol to represent summation of
all the observations.

\begin{equation}
\label{eq-abc}

 \Large \color{blue}{\bar{x}} = \frac{\sum_{i=1}^n x_i}{n}
 
 \end{equation}

{[}\code{]}

This might look intimidating, but it means the same thing as (1).


\end{document}
